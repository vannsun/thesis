\section{Problembeschreibung}
Derzeit posten die Abteilungen die verfügbaren Forschungsthemen, die als Abschlussarbeiten gelten können, in den Pinnwänden der Fakultätsbüros, sodass interessierte Studierende an der Universität sein müssen, um über diese Themen informiert zu werden.
Einige Abteilungen veröffentlichen die Themen zusätzlich auf der Website der Universität, es gibt eine zentrale Seite und da drunter Seiten für Abteilungen. Die Mitglieder der Abteilung müssen dann ständig auf die Aktualisierung der Projekte auf der Website achten aber nicht jeder Mitarbeiter hat Zugangsrechte auf die Aktualisierung des Inhalts der Website. Es ist nicht immer klar, welche Projekte bereits in Bearbeitung sind und welche noch verfügbar sind.\\

Andere Abteilungen wie Eingebettete Hardware- / Software-Systeme bitten an Studierende, die an Abschlussarbeitsthemen interessiert sind, eine formlose Bewerbung mit   den bisher belegten Modulen, persönlichen Kenntnissen und Interessen, an einen wissenschaftlichen Mitarbeiter der Abteilung zu senden. Dieser wird Ihnen dann mögliche Themen zukommen lassen und bei Bedarf einen Termin organisieren\cite{EHS:2020}.
Obwohl die Studierenden auf diese Weise eine persönliche Beratung erhalten, hängen die Antwortzeiten von der Verfügbarkeit der Mitarbeiter der Abteilung ab, und es kann sein, dass die angebotenen Projekte für den Studenten nicht von Interesse sind, was die Suche nach einem Forschungsthema weiterhin verzögern würde. Die Kommunikation erfolgt über Emails, die übersehen werden können und die nicht direkt zu einem Projekt verknüpft werden können, um die effektive Verwaltung der Abschlussarbeit zu gewährleisten.\\

Die aktuelle Art und Weise, in der die als Abschlussarbeit verfügbaren Forschungsthemen veröffentlicht werden, ermöglicht es den Studierenden nicht, eine schnelle, effiziente und personalisierte Suche entsprechend den Bedürfnissen der einzelnen Studierenden durchzuführen. Für die Mitarbeiter der Abteilung ist es derzeit nicht effizient, die Veröffentlichung verfügbarer Themen aufrechtzuerhalten und Informationsanfragen von Studenten weiterzuverfolgen.