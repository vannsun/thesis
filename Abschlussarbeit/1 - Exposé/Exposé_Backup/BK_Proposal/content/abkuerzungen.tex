\begin{longtable}[ht]{ll}
ADDIE & Analysis, Design, Development, Implementation, Evaluation\\  %E-Learning
ADEPT & Advanced Decision Environment for Process Tasks\\  %Konzept
ADL & Advanced Distributed Learning Initiative\\  %E-Learning
AGB & Allgemeine Gesch�ftsbedingungen\\  %Konzept
AGOF & Arbeitsgemeinschaft Online Forschung\\  %Einleitung
AICC & Aviation Industry Computer Based Training Committee\\  %E-Learning
API & Application Programming Interface\\  %TEL, Konzept
%ARIADNE & Alliance of Remote Instructional Authoring and Distribution Networks for Europe\\  %E-Learning
ARIS & Architektur integrierter Informationssysteme\\  %Konzept
ASC & Accredited Standards Committee\\  %EC-Grundlagen
ASP & Application Service Providing\\  %Integration
ASTD & American Society for Training and Development\\  %E-Learning
AXIS & Apache eXtensible Interaction System\\  %Implementierung
B2B & Business-to-Business\\  %Konzept, Glossar
B2C & Business-to-Consumer\\  %Konzept, Glossar
BDSG & Bundesdatenschutzgesetz\\  %EC-Grundlagen
BGB & B�rgerliches Gesetzbuch\\  %Konzept
BITKOM & Bundesverband Informationswirtschaft, Telekommunikation und neue Medien\\  % Grundlagen E-Commerce
BMBF & Bundesministerium f�r Bildung und Forschung\\  %Einleitung
BME & Bundesverband Materialwirtschaft, Einkauf und Logistik\\  %EC-Grundlagen
BPEL4WS & Business Process Execution Language for Web Services\\  %Implementierung
BSCW & Basic Support for Cooperative Work\\  %E-Learning
BSI & Bundesamt f�r Sicherheit in der Informationstechnik\\ %Literatur
CAL & Computer Aided/Assisted Learning\\  %E-Learning
CBT & Computer Based Training\\  %E-Learning, Vorgaben
CC & Creative Commons\\  %Konzept
CD & Compact Disc\\  %E-Learning
CELab & Labor f�r Content Engineering\\  %TEL
CMI & Computer Managed Instruction\\  %E-Learning
CMS & Content Management System\\  %TEL, Konzept
CORBA & Common Object Request Broker Architecture\\  %Konzept
CPU & Central Processing Unit\\  %Konzept
CSCL & Computer Supported Collaborative Learning\\  %E-Learning
CSCW & Computer Supported Cooperative Work\\  %TEL
CSS & Customer Support Services\\  %Konzept
CRM & Customer Relationship Management\\  %Konzept
CUL & Computerunterst�tztes Lernen\\  %E-Learning
DBMS & Datenbankmanagementsystem\\  %Konzept
DCOM & Distributed Component Object Model\\  %Konzept
DFN & Deutsches Forschungsnetz\\  %E-Learning
DIN & Deutsches Institut f�r Normung\\  %Lernen, E-Learning, TEL
DREL & Digital Rights Expression Language\\  %Konzept
DRM & Digital Rights Management\\  %Konzept
DVD & Digital Video Disc\\  %E-Learning
E2B & Education-to-Business\\  %Integration
E2C & Education-to-Consumer\\  %Integration
E2E & Education-to-Education\\  %Integration
EAI & Enterprise Application Integration\\  %Konzept
EAN & European Article Numbering\\  %Grundlagen-EC-Standards
EBPP & Electronic Bill Presentment and Payment\\  %Konzept
ebXML & Electronic Business Extensible Markup Language\\ %EC-Grundlagen
ECA & Event-Condition-Action\\  %Vorgaben
EC & Electronic Cash\\  %EC-Grundlagen, Konzept
ECC & E-Learning Courseware Certification\\  %E-Learning
EDI & Electronic Data Interchange\\  %EC-Grundlagen
EDIFACT & EDI for Administration, Commerce and Transport\\  %Konzept
EFQM & European Foundation for Quality Management\\  %TEL
EGBGB & Einf�hrungsgesetz zum B�rgerlichen Gesetzbuch\\  %Konzept
EITO & European Information Technology Observatory\\  % Grundlagen E-Commerce
ELAN & E-Learning Academic Network Niedersachsen\\  %Einleitung, Vorgaben
\end{longtable}
