\section{Problembeschreibung}
Derzeit posten die Abteilungen die verf\"ugbaren Forschungsthemen, die als Abschlussarbeiten gelten k\"onnen, in den Pinnw\"anden der Fakult\"atsb\"uros, sodass interessierte Studierende an der Universit\"at sein m\"ussen, um \"uber diese Themen informiert zu werden.
Einige Abteilungen ver\"offentlichen die Themen zus\"atzlich auf der Website der Universit\"at, f\"ur die die Mitglieder der Abteilung st\"andig die Aktualisierung der Projekte auf der Website anfordern m\"ussen, und es ist nicht immer klar, welche Projekte bereits in Bearbeitung sind und welche noch verf\"ugbar sind.\\

Andere Abteilungen wie Eingebettete Hardware- / Software-Systeme bitten an Studierende, die an Abschlussarbeitsthemen interessiert sind, eine formlose Bewerbung mit   den bisher belegten Modulen, pers\"onlichen Kenntnissen und Interessen, an einen wissenschaftlichen Mitarbeiter der Abteilung zu senden. Dieser wird Ihnen dann m\"ogliche Themen zukommen lassen und bei Bedarf einen Termin organisieren\cite{EHS:2020}.
Obwohl die Studierenden auf diese Weise eine pers\"onliche Beratung erhalten, h\"angen die Antwortzeiten von der Verf\"ugbarkeit der Mitarbeiter der Abteilung ab, und es kann sein, dass die angebotenen Projekte f\"ur den Studenten nicht von Interesse sind, was die Suche nach einem Forschungsthema weiterhin verz\"ogern w\"urde. Die Kommunikation erfolgt \"uber Emails, die \"ubersehen werden k\"onnen und die nicht direkt zu einem Projekt verkn\"upft werden k\"onnen, um die R\"uckverfolgbarkeit der Abschlussarbeit zu behalten.\\

Die aktuelle Art und Weise, in der die als Abschlussarbeit verf\"ugbaren Forschungsthemen ver\"offentlicht werden, erm\"oglicht es den Studierenden nicht, eine schnelle, effiziente und personalisierte Suche entsprechend den Bed\"urfnissen der einzelnen Studierenden durchzuf\"uhren. F\"ur die Mitarbeiter der Abteilung ist es derzeit nicht effizient, die Ver\"offentlichung verf\"ugbarer Themen aufrechtzuerhalten und Informationsanfragen von Studenten weiterzuverfolgen.