\subsection{Einführung}
Die Struktur des Studienplans eines akademischen Programms an der Carl von Ossietzky Universit\"at Oldenburg besteht aus vier verschiedenen Modulkategorien. Zum einen aus Basis- und Aufbaumodulen, die zu dem Pflichtbereich geh\"oren und die wichtige Grundlage des Studiums vermitteln; zum zweiten aus Akzentssetzungsmodule, die Studierenden eine individuelle Ausrichtung geben k\"onnen. Zus\"atzlich gibt es Professionalisierungs- und Praxismodule, die den Erwerb berufsbezogener und praktischer Kenntnisse stellen. Zul\"asst muss ein Abschlussmodul abgeschlossen werden, sowohl in Bachelor- als auch in Masterstudieng\"angen vorhanden.
Das Abschlussmodul ist eine vertiefende \"Ubung, die von den Studenten als Voraussetzung f\"ur die Qualifikation des Abschlusses entwickelt wurde\cite{BScInf:2020}.\\

Bei der Anfertigung der Abschlussarbeit muss die\textbackslash der Studierende zun\"achst ein Thema f\"ur ihre\textbackslash seine Forschungsarbeit finden, welches f\"ur sein Studienprogramm relevant sein muss. Es ist m\"oglich, dass Studierende in Absprache mit einer Gutachterin bzw. einem Gutachter selbst ein Thema f\"ur eine Abschlussarbeit w\"ahlen k\"onnen, oder dass, es durch eine Einrichtung au{\ss}erhalb der Universit\"at ausgef\"uhrt werden kann\cite{Boles:2015}.\\

Das Department f\"ur Informatik ist in vier Fachrichtungen aufgegliedert: Theoretische Informatik, Praktische Informatik, Angewandte Informatik und Technische Informatik, und jede davon ist nochmal in verschiedene Vertiefungsrichtungen unterteil. Insgesamt gibt es siebzehn Abteilungen\cite{SpeInf:2020}. Jede Abteilung setz sich aus Professoren und wissenschaftliche Mitarbeiter zusammen, die eine Forschungsgruppe bilden und f\"ur die von ihr durchgef\"uhrten Projekte verantwortlich sind.
F\"ur Abschlussprojekte, die an der Universit\"at entwickelt wurden, jede Abteilung des Departments f\"ur Informatik sollte, auf einer Abteilungs-eigenen Website, aktuelle Themen f\"ur Abschlussarbeiten ver\"offentlichen oder beschreiben, wie Studierende in der entsprechenden Abteilung ein Thema erh\"alt\cite{Boles:2015}.\\

Der Zugang zu Informationen \"uber die Abschlussthemen sollte von dem Department f\"ur Informatik gew\"ahrleistet werden, indem es eine standardisierte Form der Suche anbietet, welche den Universit\"atsmitarbeitern erm\"oglicht, die ver\"offentlichten Informationen effizient zu aktualisieren, und Studierende erm\"oglicht, ihre pers\"onlichen Interessen zu ber\"ucksichtigen.

\newpage
\subsubsection{Zielsetzung}
Diese Bachelorarbeit besch\"aftigt sich gezielt mit der Themenfindung und R\"uckverfolgbarkeit von Abschlussarbeiten. In diesem Zusammenhang sollte klargestellt werden, welche Kriterien sind f\"ur die Suche nach einem Forschungsthema relevant und unter welchen Parametern kann der aktuelle Prozess der Ver\"offentlichung und Zuordnung von Abschlussarbeitsthemen optimiert werden. Diese Informationen werden f\"ur die Entwicklung und Implementierung einer Softwareanwendung ber\"ucksichtigt, mit der Studierende nach Themen von Abschlussarbeiten suchen und Mitarbeiter des Fachbereichs Informatik diese zuweisen und verfolgen k\"onnen.
