\documentclass[11pt,a4paper]{book} % Basisdokumentenklasse
\usepackage[latin1]{inputenc}      % F\"{u}r Umlaute
\usepackage[T1]{fontenc}           % und {\ss}
\usepackage[ngerman]{babel}        % Deutsche Standardbezeichner und Trennung
\usepackage{makeidx}               % Indexpaket
\usepackage{times}                 % Sch\"{o}nere Schriften, Achtung, nicht pslatex verwenden!
\usepackage{longtable}             % Lange Tabelle f\"{u}r Abk\"{u}rzungsverzeichnis
\usepackage{graphics}              % EPS-Grafiken einbinden
\usepackage{hyperref}
%\usepackage{enumitem}

\usepackage{fancyvrb}              % Fancy-Verbatin f\"{u}r Programmlistings

\usepackage{style/abtisstud}           % Das Abteilungs-Stylepaket, am Ende einbinden!

% Einstellungen f\"{u}r die "offizielle" Titelseite
\title{Entwicklung eines Stud.IP-Plugins\\[0.2em] zur Themenverwaltung von Abschlussarbeiten} % Titel der Arbeit
\author{Cinddy Vannessa Ca�on Pasquel} % Name des Autors \~{n}
\date{\today}
\studiengang{Bachelorstudiengang Informatik}
\arbeitstyp{Bachelorarbeit}
\erstgutachter{Dr. Ute Vogel-Sonnenschein}
\zweitgutachter{Dipl. Nico M�ller}

% Index erneuern
\makeindex

%\includeonly{content/kapitel1}

\setcounter{tocdepth}{5}
\setcounter{secnumdepth}{5}
% Das eigentliche Dokument
\begin{document}
%Damit die Hyperrefs richtig funktionieren (vor allem bei Seite 1)
\pagenumbering{gobble}

% Anfang des Dokuments
\maketitle              % Offizielle Titelseite der Uni

\newpage
\thispagestyle{empty}
%\cleardoublepage

\pagenumbering{Roman}   % R\"{o}mische Seitenzahlen f\"{u}r den Anfang

% Zusammenfassung und Abstract
\chapter*{Zusammenfassung}
Dieses Dokument repr�sentiert Struktur und Designvorlage des neuen
Abteilungsstils f�r studentische Arbeiten.



% Inhaltsverzeichnis
\renewcommand{\contentsname}{Inhalt} % Es soll nicht Inhaltsverzeichnis hei{\ss}en, sondern Inhalt
\tableofcontents        % Inhaltsverzeichnis einbinden
\cleardoublepage        % Danach auf ungerader Seite weitermachen
\pagenumbering{arabic}  % Arabische Seitenzahlen starten neu


% Der eigentliche Inhalt, hier k\"{o}nnte man dann auch jedes Kapitel einzeln einf\"{u}gen
% KAPITEL 1
\chapter{Einleitung} \index{Einleitung}
\section{Motivation}

Die Struktur des Studienplans eines akademischen Bachelor an der Carl von Ossietzky Universit�t Oldenburg besteht aus vier verschiedenen Modulkategorien. Zum einen aus Basis- und Aufbaumodulen, die zu dem Pflichtbereich geh�ren und die wichtige Grundlage des Studiums vermitteln; zum zweiten aus Akzentssetzungsmodule, die Studierenden eine individuelle Ausrichtung geben k�nnen. Zus�tzlich gibt es Professionalisierungs- und Praxismodule, die den Erwerb berufsbezogener und praktischer Kenntnisse stellen. Zul�sst muss ein Abschlussmodul abgeschlossen werden.
Master of Education und Master sind anderes aufgebaut aber auch f�r sie gibt es Abschlussarbeit.\\
Das Abschlussmodul ist eine vertiefende Pr�fungsleitung, die von den Studenten als Voraussetzung f�r die Qualifikation des Abschlusses entwickelt wurde\cite{BScInf:2020}.\\

Bei der Anfertigung der Abschlussarbeit muss die\textbackslash der Studierende zun�chst ein Thema f�r ihre\textbackslash seine Forschungsarbeit finden, welches f�r sein Studienprogramm relevant sein muss. Es ist m�glich, dass Studierende in Absprache mit einer Gutachterin bzw. einem Gutachter selbst ein Thema f�r eine Abschlussarbeit w�hlen k�nnen, oder dass, es durch eine Einrichtung au{\ss}erhalb der Universit�t ausgef�hrt werden kann\cite{Boles:2015}.\\

Das Department f�r Informatik ist in vier Fachrichtungen aufgegliedert: Theoretische Informatik, Praktische Informatik, Angewandte Informatik und Technische Informatik, und jede davon ist nochmal in verschiedene Vertiefungsrichtungen unterteil. Insgesamt gibt es zur Zeit siebzehn Abteilungen\cite{SpeInf:2020}. Jede Abteilung setz sich aus Professoren und wissenschaftliche Mitarbeiter zusammen, die eine Forschungsgruppe bilden und f�r die von ihr durchgef�hrten Projekte verantwortlich sind.\\
F�r Abschlussprojekte, die an der Universit�t entwickelt wurden, sollte jede Abteilung des Departments f�r Informatik auf einer Abteilungs-eigenen Website, aktuelle Themen f�r Abschlussarbeiten ver�ffentlichen oder beschreiben, wie Studierende in der entsprechenden Abteilung ein Thema erh�lt\cite{Boles:2015}.\\

Der Zugang zu Informationen �ber die Abschlussthemen sollte von dem Department f�r Informatik gew�hrleistet werden, indem es eine standardisierte Form der Suche anbietet, welche den Universit�tsmitarbeitern erm�glicht, die ver�ffentlichten Informationen effizient zu aktualisieren, und Studierende erm�glicht, ihre pers�nlichen Interessen zu ber�cksichtigen. Die aktuelle L�sung, alle Abschlussthemen auf einer zentralen Website zu verkn�pfen, hat sich nicht bew�hrt und sollte durch diese Arbeit abgel�st werden.

Die Suche nach Abschlussthemen durch Studierende ist nicht einfach.  Das umfangreiche Angebot an Spezialisierungen bzw. Abteilungen des Departments f�r Informatik erm�glicht es den Studenten, sich mit dem Bereich zu befassen, der sie am meisten anspricht. Es gibt jedoch zahlreiche Forschungsthemen, die sogar zu verschiedenen Studienbereichen geh�ren k�nnen, wie bei Projekten der Abteilung Didaktik der Informatik - Fachgebiet Angewandte Informatik, wo Kenntnisse der Mikrorobotik / Regelungstechnik und Softwaretechnik, die Teil der Fachrichtungen Technische Informatik bzw. Praktische Informatik sind, erforderlich sein k�nnen.\\
\include{content/kapitel1/1_problembeschreibung}
\include{content/kapitel1/2_stand}
\include{content/kapitel1/3_aufgabenstellung}


% KAPITEL 2
\include{content/kapitel2/0_ausgangssituation}
\section{Struktur des Departements f�r Informatik}\index{Struktur}\index{Departament}
Das Department geh�rt zur Fakult�t II - Informatik, Wirtschafts- und Rechtswissenschaften, es besteht derzeit aus

\begin{itemize}
	\item[--] 17 ProfessorInnen und ihren Abteilungen17 ProfessorInnen und ihren Abteilungen
	\item[--] ca.1400 Studierenden, ca. 50 wissenschaftlichen Mitarbeitern
	\item[--] drei weiteren f�r das Department zentralen Einrichtungen, die Service-Aufgaben in Lehre und Forschung �bernehmen
	\item[--] sowie einer Departementsverwaltung
\end{itemize}

Das Department f�r Informatik ist in vier Fachrichtungen aufgegliedert und jede davon ist nochmal in verschiedene Vertiefungsrichtungen unterteilt.

\begin{itemize}
	\item Theoretische Informatik
	\begin{itemize}
		\item Entwicklung korrekter Systeme
		\item Parallele Systeme
	\end{itemize}
	\item Praktische Informatik
	\begin{itemize}
		\item Medieninformatik und Multimedia-Systeme
		\item Computational Intelligence
		\item Systemsoftware und verteilte Systeme
		\item Softwaretechnik
		\item Informationssysteme
	\end{itemize}
	\item Angewandte Informatik
	\begin{itemize}
		\item Didaktik der Informatik
		\item Wirtschaftsinformatik Systemanalyse und Optimierung
		\item Intelligente Transportsysteme
		\item Energieinformatik
		\item Wirtschaftsinformatik Very Large Business Applications
		\item Digitalisierte Energiesysteme
	\end{itemize}
	\item Technische Informatik
	\begin{itemize}
		\item Sicherheitskritische eingebettete Systeme
		\item Mikrorobotik und Regelungstechnik
		\item Hybride Systeme
		\item Eingebettete Hardware-/Software-Systeme
	\end{itemize}
\end{itemize}

Jede Abteilung setz sich aus Professoren und wissenschaftliche Mitarbeiter zusammen, die eine Forschungsgruppe bilden und f�r die von ihr durchgef�hrte Projekte verantwortlich sind.
F�r Abschlussprojekte, die an der Universit�t entwickelt wurden, sollte im Moment jede Abteilung des Departments f�r Informatik auf einer Abteilungs-eigenen Webseite, aktuelle Themen f�r Abschlussarbeiten ver�ffentlicht oder beschreiben, wie Studierende in der entsprechenden Abteilung ein Thema erhalten kann.
\section{Technische Grundlagen der Software}\index{Technische}\index{Grundlagen}
Dieses Unterkapitel zielt darauf ab, den \textbf{Begriff Weiterentwicklung von Softwaresystemen} n�her zu betrachten. //TODO: Beschreibung Kapitel

\subsection{Weiterentwicklung von Softwaresystemen}\index{Weiterentwicklung}\index{Softwaresysteme}

\subsection{LimeSurvey - Umfrage Tool}\index{Weiterentwicklung}\index{Softwaresysteme}


\subsection{Objektorientierte Programmierung}\index{Objektorientierte Programmierung}\index{Softwaresysteme}


\subsection{Design Patterns}\index{Design Patterns}


\subsubsection{Model View Controller}\index{Objektorientierte Programmierung}\index{Softwaresysteme}


\subsection{Entwicklungssystem mit XAMPP}\index{XAMPP}
XAMPP ist ein Open-Source-Softwarepaket, der das Installieren und Konfigurieren des Webservers Apache mit dem Datenbankverwaltungssystem MariaDB und die Skriptsprachen PHP und Perl erm�glicht. 
Der Name ist ein Akronym: \textbf{X} (f�r eines der verschiedenen Betriebssysteme cross-platform), \textbf{A}pache, \textbf{M}ariaDB / MySQL, \textbf{P}HP, \textbf{P}erl. Ab Version 5.6.15 hat XAMPP die MySQL-Datenbank in MariaDB ge�ndert, eine GPL-lizenzierte Abzweigung von MySQL.

Die Entwicklungsumgebung wurde durch Installation von Version 7.3.23 von XAMPP f�r das Betriebssystems Windows 10 konfiguriert.

Bei der Durchf�hrung dieses Projekts wurden Software-Artefakte mit MariaDB und PHP generiert.

\subsection{MariaDB}\index{MariaDB}\index{MySQL}


Die Entwicklungsgeschichte von MySQL geht bis ins Jahr 1979 zur�ck, bis es schlie�lich von der Firma Oracle �bernommen wurde. 

Die Verwendungsspektrum von MariaDB ist gro�, da es alle vorteile von kommerziellen Anbietern verenint. MariaDB setzt dabei auf die klassische Cliente-Server-Architektur, in der ein zentraller Datenbankserver die Daten verwaltet, worauf ein Datenbank-Client �ber das netzwerzugreifen kann. Weitere Vorteile sind die meherbenutzerf�higkeit ohne Performance-Einbu�en, die l


\subsection{PHP}\index{PHP}
PHP existiert bereits seit dem Jahre 1994 und war anfangs sehr beschr�nkt in seinem Funktionsumfang.
Zun�chst war es lediglich f�r die Dynamisierung von statischen Webseiteninhalten geeignet.
Erst im Jahre 1998 wurde die Funktionalit�t durch Version 3.0 deutlich erweitert, da von nun an
verschiedene Datenbanksysteme angebunden werden konnten.

Im Laufe der Zeit wurde PHP stetig weiterentwickelt und steht zum gegenw�rtigen Zeitpunkt in der
Version 5.5.15 zur Verf�gung.

Zwar wird PHP von vielen Entwicklern aufgrund der Einfachheit sehr kritisch bewertet, es steht
gegen�ber den h�heren Programmiersprachen, wie Java oder C++, in nichts nach. Dem Entwickler /
Entwicklerin stehen beispielsweise Konzepte wie Kapselung, Vererbung, Namensr�ume genauso zur
Verf�gung wie Closures, womit Funktionalit�ten in anonyme Methoden gekapselt werden k�nnen. Somit ist das objektorientierte Programmieren im vollen Umfang m�glich.

Ein weiterer Vorteil von PHP ist die Plattformunabh�ngigkeit, wodurch es auf den meisten Betriebssystemen installiert und genutzt werden kann. Anzumerken ist jedoch auch, dass PHP nur eine Skriptsprache ist und erst zur Laufzeit vom PHP-Interpreter ausgef�hrt wird. Das hat zwar den Vorteil, dass eine �bersetzung in Maschinencode nicht n�tig ist, ist auch ein Nachteil bei zeitkritischen Aufgaben.

Ein weiterer, nicht neuer Ansatz soll die Vielseitigkeit von PHP noch einmal verdeutlichen: Facebook
transformiert den PHP-Quellcode in ein �quivalentes C++ Konstrukt und �bersetzt dies mit einem geeigneten.

Compiler in Maschinencode. Der Vorteil dabei ist, dass jene Anwendung einerseits schneller
ist und andererseits die Webserver entlastet.
PHP erfreut sich gro�er Beliebtheit, da mittlerweile mehr als 240 Millionen Webseiten PHP f�r die
Erstellung von dynamischen Webseiten verwenden.
\include{content/kapitel2/3_LMS}
\section{Stud.IP}\index{Stud.IP}
Stud.IP ist ein Campus-Lernmanagementsystem und steht f�r "Studienbegleitender Internetsupport von Pr�senzlehre" \cite{stIP20}. Alle Unterrichtsprozesse werden von Stud.IP unterst�tzt, indem die Verwaltung der Prozesse von der Planung der Unterrichten und Zuordnung der Unterrichtsr�ume bis zum Qualit�tsmanagement konsolidiert wird. 

Stud.IP besteht aus einer Kernumgebung, die eine Vielzahl von Tools anbietet, von der kollaborativen Texterstellung �ber Lernmodule bis hin zum Dokumentenmanagement, es werden im Kern Basisfunktionalit�ten bereitgestellt. Der Kern von Stud.IP kann weiterentwickelt werden um die Funktionalit�ten des Systems zu erweitern.

Stud.IP und die meisten der verf�gbare Plugins sind Open Source Software und stehen unter der General Public License. Der Kern ist nachhaltig und wird kontinuierlich weiterentwickelt.

Die Universit�t Oldenburg implementierte zum Wintersemester 2003/2004 das Open-Source-LMS Stud.IP und wechselte vom kommerziellen LMS Blackboard \cite{appelrath06}. Einer der Vorteile des Open-Source-Ansatzes besteht darin, dass lokale Anpassungen an den Anforderungen der Lehrenden und Studierenden vorgenommen werden k�nnen.
F�r die Universit�t Oldenburg wurden einige Module eingebaut wie die Gruppenverwaltung in Veranstaltungen oder das "Schwarzes Brett" Plugin.

\subsection{Softwarearchitektur}

Stud.IP ist eine PHP-Softwareanwendung, die eine MySQL-Datenbank verwendet. Als Webserver wird der Apache unterst�tzt. 

Wer mitentwickeln will, braucht also vor allem PHP-Kenntnisse, muss sich etwas mit SQL auskennen und ein bisschen �ber Apache-Konfiguration wissen. Und, wie immer bei Webanwendungen: Alle Ausgaben geschehen in HTML, formatiert duch CSS. Einige Funktionen verwenden zudem XML als Zwischenformat, Javascript und AJAX sind ebenfalls an vielen Stellen pr�sent. Wenn all das keine Fremdw�rter f�r dich sind, bis du gut ger�stet.

%\subsection{Trails}
\subsection{Stud.IP Datenbank}
Bei der Stud.IP-Datenbank handelt es sich um eine MariaDB-Datenbank. Das Datenbankschema ist �ber die Jahre erweitert worden, es sind keine Informationen �ber die Struktur zu finden.


% KAPITEL 3
\chapter{Anforderungsanalyse} \index{Anforderungsanalyse}
% REVISAR
Eine Anforderungsanalyse hilft die Erwartungen eines Auftraggebers an ein Softwareprojekt
zu verstehen. Diese Analyse erm�glicht die Ermittlung aller Anforderung, die f�r die
Entwicklung des neuen Systems ber�cksichtigt werden([14], S.55).
In diesem Abschnitt wird zun�chst das Ergebnis Analyse der aktuellen Informationen zu
Abschlussarbeiten am Departement f�r Informatik vorgestellt. Dann werden alle Stakeholder
des k�nftigen Systems dargelegt. Schlie�lich werden die Anforderungen und die
Akteure des geplanten Systems beschrieben. Abschlie�end werden die Priorit�ten dieser
Anforderungen aufgesetzt.

Damit ...das resultierende System... den Erwartungen der Lehrenden und Studierenden so genau wie m�glich entsprechen,
wurde eine Umfrage durchgef�hrt.
Die Online Plattform LimeSurvey wurde f�r als Tool f�r die Umfrage ausgew�hlt, da es f�r die Universit�t Oldenburg lizenziert ist, bietet verschiedene Fragetypen an und erm�glicht auch die Daten graphisch zu analysieren.
\section{Umfrage}



\section{Umfrage f�r die Lehrende}

\section{Umfrage f�r die Studierende}

\subsection{Ergebnisse der Umfrage und Datenanalyse}
\section{User Stories}\index{User Stories}
\section{Funktionale Anforderungen}\index{Funktionale Anforderungen}
\section{Nicht Funktionale Anforderungen}\index{Nicht Funktionale Anforderungen}
\section{Definition der Suchkriterien und Suchalgorithmus}\index{Suchkriterien}
\include{content/kapitel3/6_modellierung}

% KAPITEL 4
\include{content/kapitel4/0_entwurf_implement}
\include{content/kapitel4/1_entscheidung}
\include{content/kapitel4/2_entwurf}
\include{content/kapitel4/3_implementierung}
\include{content/kapitel4/4_tests}
\include{content/kapitel4/5_ergebnisse}

\chapter{Anwendung des Style-Pakets} \index{Anwendung}
Nachfolgend ein kurzer Leitfaden f\"{u}r die Anwendung des
vorliegenden Style-Pakets f\"{u}r LaTeX-Dokumente f\"{u}r Arbeiten in der
Abteilung Informationssysteme

\begin{itemize}
\item Autor: Markus Schmees
\item Anpassungen f\"{u}r studentische Arbeiten: Marco Grawunder und Richard Hackelbusch
\item Version: 1.3.2
\item Stand: 24.04.2009
\end{itemize}

\section{Einbinden des Pakets}\index{Einbinden}\index{Paket}
Die einfachste M\"{o}glichkeit besteht darin, das vorgegebene Dokument
mit dem Titel

\texttt{BEARBEITER.tex}

zu verwenden und um eigene
Inhalte zu erg\"{a}nzen. BEARBEITER sollte durch die Art der Arbeit und den Namen des Autors ersetzt werden. Alternativ dazu kann der Autor im eigenen
\LaTeX-Hauptdokument, das die Dokumentenklasse \texttt{book}, eine
\texttt{11pt}-Schrift sowie die Blattgr\"{o}{\ss}e \texttt{a4paper}
verwenden muss, zus\"{a}tzlich am Ende der Paketdeklarationen das
Abteilungsstylepaket durch folgenden Befehl einbinden

\begin{verbatim}
\usepackage{abtisstud/abtisstud}
\end{verbatim}

Hierbei aber darauf achten, dass das Paket als \textbf{letztes
Paket} eingebunden wird, da es diverse \LaTeX-Befehle
\"{u}berschreibt, die von anderen Paketen ebenfalls \"{u}berschrieben
werden k\"{o}nnen. Das resultierende Hauptdokument sollte dann die in
der folgenden Abb.~\ref{stylepaket} dargestellte Struktur besitzen

\begin{figure}[ht]
\centering
\begin{Verbatim}[label=dissertation.tex,numberblanklines=false,fontsize=\scriptsize,numbers=left,frame=single]
\documentclass[11pt,a4paper]{book} % Basisdokumentenklasse
\usepackage[latin1]{inputenc}      % F\"{u}r Umlaute
\usepackage[T1]{fontenc}           % und {\ss}
\usepackage[ngerman]{babel}        % Deutsche Standardbezeichner und Trennung
\usepackage{makeidx}               % Indexpaket
\usepackage{times}                 % Sch\"{o}nere Schriften
\usepackage{longtable}             % Lange Tabelle f\"{u}r Abk\"{u}rzungsverzeichnis
\usepackage{graphics}              % EPS-Grafiken einbinden
\usepackage{abtisstud/abtisstud}   % Das Stylepaket am Ende einbinden!

% Das eigentliche Dokument
\begin{document}
... Inhalt ...
\end{document}
\end{Verbatim}
\caption{Einbinden des Style-Pakets}\label{stylepaket}
\end{figure}


\section{\"{A}nderungen gegen\"{u}ber \texttt{book}}\index{Anderungen@\"{A}nderungen}
Dem Erweiterungspaket zugrunde liegt der Dokumentenstil
\texttt{book}. Dieser wurde mit Hilfe des vorliegenden Pakets an
Besonderheiten der IS-Formatvorlage angepasst. Dabei wurden
folgende Dinge ver\"{a}ndert:

\begin{itemize}
\item Abschnittnummerierung auf Ebene 3 gesetzt (bis 1.1.1.1)
\item Auff\"{u}hrung im Inhaltsverzeichnis bis Ebene 1 (bis 1.1)
\item R\"{a}nder, Header, Textweite und Texth\"{o}he angepasst
\item Linken Einzug um 0,37cm bei neuem Absatz eingef\"{u}gt
\item Abstand der Abs\"{a}tze auf 6pt eingestellt
\item Seitenzahlen und Abschnitte im Header dargestellt
\item Abstand der \"{U}berschriften zum Text eingestellt
\item Farbe und Schrift der \"{U}berschriften festgelegt
\item Verzeichnisdeklarationen f\"{u}r korrekte Headeranzeige
\"{u}berschrieben
\item Abstand der Abbildungen zu den Unterschriften angepasst
\item Schriftart und Style f\"{u}r Abbildungsunterschriften
eingestellt
\item "`Offizielle"' Titelseite integriert
\item Seitenzahltypen (r\"{o}misch und arabisch) kombiniert
\item Fu{\ss}noten linksb\"{u}ndig und mit Abstand zwischen Nummer und
Text
\item S\"{a}mtliche Listen linksb\"{u}ndig
\item Geringere Einr\"{u}ckung der Unterstichpunkte im Index
\item Umwandlung der \"{U}berschriften f\"{u}r Verzeichnisse in
gleichwertige \"{U}berschriften ohne "`-ver\-zeich\-nis"'
\end{itemize}


\section{Einbinden von Abbildungen}\index{Abbildungen}
\index{Abbildungen}
\index{Einbinden!Abbildungen}
Wie bei der normalen Verwendung von \LaTeX{} k\"{o}nnen auch hier \"{u}ber
die entsprechenden Zusatzpakete Grafiken eingebunden werden. Immer
darauf achten, dass diese als Vektorgrafiken vorliegen, damit sie
problemlos skaliert und damit auch ohne Qualit\"{a}tsverlust gedruckt
werden k\"{o}nnen. Nachfolgend zeigt Abb.~\ref{abbildung} am Beispiel
des Uni-Logos, wie eine Abbildung eingebunden und angezeigt werden
kann (vlg. dazu auch Abschnitt \ref{Schriften}).

\begin{figure}[ht]
  \centering
  \includegraphics[scale=1]{style/unilogo.pdf}
  \caption{Eine sch\"{o}ne Abbildung}\label{abbildung}
\end{figure}


\section{Erstellen und Anpassen des Indexes}\index{Indexerstellung}
Um das grundlegende Indexfile zu erzeugen, muss der Befehl
\texttt{makeindex} innerhalb des Haupt-\LaTeX-Dokuments aufgef\"{u}hrt sein.
Dieser erzeugt aus s\"{a}mtlichen \texttt{index}-Befehlen die Datei

\texttt{dissertation.idx}.

Um diese Datei ebenfalls passend zu
formatieren, wurde ein entsprechendes Stylefile namens

\texttt{abtisstud.ist}

erstellt. Die Anwendung dieses Styles auf den
Index erfolgt mit Hilfe des folgenden Befehls

\begin{verbatim}
makeindex -s abtisstud/abtisstud.ist -o content/index.tex \
    dissertation.idx
\end{verbatim}

Dieser Befehl sorgt daf\"{u}r, dass die Indexdaten entsprechend
formatiert und in die Datei \texttt{index.tex} in das Verzeichnis
\texttt{content} abgelegt werden. Von dort kann der neue Index
dann problemlos per input eingelesen werden. Also bitte nicht den
traditionell erzeugten Index \"{u}ber den Befehl
\texttt{printindex} einbinden! Der Auszug aus dem
\LaTeX-Hauptdokument zur Erzeugung und Einbindung des Indexes sieht
folgenderma{\ss}en aus

\begin{figure}[ht]
\centering
\begin{Verbatim}[label=dissertation.tex,numberblanklines=false,fontsize=\scriptsize,numbers=left,frame=single]
...

% Indexdatei erzeugen
\makeindex

% Das eigentliche Dokument
\begin{document}

...

\index{Irgendwas}                     % Einstufiger Indexeintrag
\index{Irgendwas!Ganzanderes}         % Zweistufiger Indexeintrag

...

% Index als letztes Verzeichnis
\cleardoublepage                      % Auf ungerader Seite beginnen
\addcontentsline{toc}{chapter}{Index} % Index im Inhaltsverzeichnis anzeigen
\begin{theindex}
{\indexfrontskip\Large\sffamily\bfseries\hfill A\hfill}\nopagebreak
 
  \item Abbildungen\dotfill 2
  \item \"Anderungen\dotfill 2
  \item Anwendung\dotfill 1

  \indexspace
{\indexfrontskip\Large\sffamily\bfseries\hfill E\hfill}\nopagebreak
 
  \item Einbinden\dotfill 1
    \subitem Abbildungen\dotfill 2

  \indexspace
{\indexfrontskip\Large\sffamily\bfseries\hfill F\hfill}\nopagebreak
 
  \item Festlegungen\dotfill 7

  \indexspace
{\indexfrontskip\Large\sffamily\bfseries\hfill I\hfill}\nopagebreak
 
  \item Indexerstellung\dotfill 3
  \item Internationalisierung\dotfill 4

  \indexspace
{\indexfrontskip\Large\sffamily\bfseries\hfill L\hfill}\nopagebreak
 
  \item Literatur\dotfill 7
  \item Literaturreferenzen\dotfill 7

  \indexspace
{\indexfrontskip\Large\sffamily\bfseries\hfill P\hfill}\nopagebreak
 
  \item Paket\dotfill 1

  \indexspace
{\indexfrontskip\Large\sffamily\bfseries\hfill S\hfill}\nopagebreak
 
  \item Schriften\dotfill 5

  \indexspace
{\indexfrontskip\Large\sffamily\bfseries\hfill T\hfill}\nopagebreak
 
  \item Titelseite\dotfill 8

  \indexspace
{\indexfrontskip\Large\sffamily\bfseries\hfill V\hfill}\nopagebreak
 
  \item Verzeichnisreihenfolge\dotfill 7

  \indexspace
{\indexfrontskip\Large\sffamily\bfseries\hfill Z\hfill}\nopagebreak
 
  \item Zitate\dotfill 7

\end{theindex}
             % Formatierten Index einbinden
\end{document}
\end{Verbatim}
\caption{Index erstellen, formatieren und integrieren}\label{indexintegration}
\end{figure}


\section{Internationalisierung}\index{Internationalisierung}
Durch Einbinden entsprechender Sprachpakete wird \LaTeX{} an
Besonderheiten bestimmter Sprachr\"{a}ume, z.B. deren Trennung oder
Verzeichnisbezeichnung angepasst. Durch die individuelle
Gestaltung bestimmter Verzeichnisnamen wurden die zugeh\"{o}rigen
Befehle aber \"{u}berschrieben. Das bedeutet nun, dass der vorliegende
Stil zwar problemlos f\"{u}r deutsche Arbeiten anwendbar ist, aber
diese Verzeichnisbezeichnung f\"{u}r Dissertationen in fremder Sprache
angepasst werden m\"{u}ssen. Diese betreffen also insbesondere die
folgenden Paket- und Verzeichnisdeklarationen in den Zeilen 3, 11,
14, 18, 19, 23, 25, 29 und 30.

\begin{figure}[htp]
\centering
\begin{Verbatim}[label=dissertation.tex,numberblanklines=false,fontsize=\scriptsize,numbers=left,frame=single]
\documentclass[11pt,a4paper]{book}

...

\usepackage[ngerman]{babel}

...

\usepackage{abtisstud/abtisstud}

...

% Das eigentliche Dokument
\begin{document}

...

% Verzeichnisse am Ende, erst das Glossar
\addonchapter{Glossar} %
Nachfolgend sind noch einmal wesentliche Begriffe dieser Arbeit
zusammengefasst und erl�utert. Eine ausf�hrliche Erkl�rung findet
sich jeweils in den einf�hrenden Abschnitten sowie der jeweils
darin angegebenen Literatur. Das im Folgenden im Rahmen der
Erl�uterung verwendete Symbol \this bezieht sich jeweils auf den
im Einzelnen vorgestellten Begriff, das Symbol \siehe{} verweist
auf einen ebenfalls innerhalb dieses Glossars erkl�rten Begriff.

\begin{description}
\item[Auktion] Eine \this ist das im \siehe{E-Commerce} am
H�ufigsten eingesetzte Verfahren zur dynamischen
\siehe{Preisfindung}. Interessenten k�nnen dabei durch Abgabe von
Geboten Preis, Dauer und Gewinner beeinflussen. Bei einer offenen
\this sind Bieter, H�he der Gebote und der aktuelle Preis f�r
alle Teilnehmer sichtbar, bei der geschlossenen (sealed) \this
erfolgt nur eine interne Benachrichtigung. Die bekanntesten Typen
sind die traditionelle \siehe{Versteigerung} sowie die
\siehe{holl�ndische}, \siehe{umgekehrte} und \siehe{verdeckte} \this.

\item[Behaviorismus] Der \this ist eine \siehe{Lerntheorie}, die
davon ausgeht, dass Wissen als Struktur unabh�ngig vom
\siehe{Lernenden} existiert und dass sein Verhalten operant
konditioniert ist, d.h. dass es als Konsequenz aus anderen
Verhaltensweisen resultiert. Erfolgt eine positive Reaktion,
beh�lt der \siehe{Lernende} neu erlerntes Verhalten bei, negative
Reaktionen f�hren zu einer Verminderung dieses Verhaltens. Der
\siehe{Lehrende} bestimmt dabei das zu erlernende Wissen und
ist f�r die Steuerung des \siehe{Lernprozesses} zust�ndig.
\end{description}


% Dann die Abk\"{u}rzungen
\addonchapter{Abk\"{u}rzungen}
\begin{longtable}[ht]{ll}
ADDIE & Analysis, Design, Development, Implementation, Evaluation\\  %E-Learning
ADEPT & Advanced Decision Environment for Process Tasks\\  %Konzept
ADL & Advanced Distributed Learning Initiative\\  %E-Learning
AGB & Allgemeine Gesch�ftsbedingungen\\  %Konzept
AGOF & Arbeitsgemeinschaft Online Forschung\\  %Einleitung
AICC & Aviation Industry Computer Based Training Committee\\  %E-Learning
API & Application Programming Interface\\  %TEL, Konzept
%ARIADNE & Alliance of Remote Instructional Authoring and Distribution Networks for Europe\\  %E-Learning
ARIS & Architektur integrierter Informationssysteme\\  %Konzept
ASC & Accredited Standards Committee\\  %EC-Grundlagen
ASP & Application Service Providing\\  %Integration
ASTD & American Society for Training and Development\\  %E-Learning
AXIS & Apache eXtensible Interaction System\\  %Implementierung
B2B & Business-to-Business\\  %Konzept, Glossar
B2C & Business-to-Consumer\\  %Konzept, Glossar
BDSG & Bundesdatenschutzgesetz\\  %EC-Grundlagen
BGB & B�rgerliches Gesetzbuch\\  %Konzept
BITKOM & Bundesverband Informationswirtschaft, Telekommunikation und neue Medien\\  % Grundlagen E-Commerce
BMBF & Bundesministerium f�r Bildung und Forschung\\  %Einleitung
BME & Bundesverband Materialwirtschaft, Einkauf und Logistik\\  %EC-Grundlagen
BPEL4WS & Business Process Execution Language for Web Services\\  %Implementierung
BSCW & Basic Support for Cooperative Work\\  %E-Learning
BSI & Bundesamt f�r Sicherheit in der Informationstechnik\\ %Literatur
CAL & Computer Aided/Assisted Learning\\  %E-Learning
CBT & Computer Based Training\\  %E-Learning, Vorgaben
CC & Creative Commons\\  %Konzept
CD & Compact Disc\\  %E-Learning
CELab & Labor f�r Content Engineering\\  %TEL
CMI & Computer Managed Instruction\\  %E-Learning
CMS & Content Management System\\  %TEL, Konzept
CORBA & Common Object Request Broker Architecture\\  %Konzept
CPU & Central Processing Unit\\  %Konzept
CSCL & Computer Supported Collaborative Learning\\  %E-Learning
CSCW & Computer Supported Cooperative Work\\  %TEL
CSS & Customer Support Services\\  %Konzept
CRM & Customer Relationship Management\\  %Konzept
CUL & Computerunterst�tztes Lernen\\  %E-Learning
DBMS & Datenbankmanagementsystem\\  %Konzept
DCOM & Distributed Component Object Model\\  %Konzept
DFN & Deutsches Forschungsnetz\\  %E-Learning
DIN & Deutsches Institut f�r Normung\\  %Lernen, E-Learning, TEL
DREL & Digital Rights Expression Language\\  %Konzept
DRM & Digital Rights Management\\  %Konzept
DVD & Digital Video Disc\\  %E-Learning
E2B & Education-to-Business\\  %Integration
E2C & Education-to-Consumer\\  %Integration
E2E & Education-to-Education\\  %Integration
EAI & Enterprise Application Integration\\  %Konzept
EAN & European Article Numbering\\  %Grundlagen-EC-Standards
EBPP & Electronic Bill Presentment and Payment\\  %Konzept
ebXML & Electronic Business Extensible Markup Language\\ %EC-Grundlagen
ECA & Event-Condition-Action\\  %Vorgaben
EC & Electronic Cash\\  %EC-Grundlagen, Konzept
ECC & E-Learning Courseware Certification\\  %E-Learning
EDI & Electronic Data Interchange\\  %EC-Grundlagen
EDIFACT & EDI for Administration, Commerce and Transport\\  %Konzept
EFQM & European Foundation for Quality Management\\  %TEL
EGBGB & Einf�hrungsgesetz zum B�rgerlichen Gesetzbuch\\  %Konzept
EITO & European Information Technology Observatory\\  % Grundlagen E-Commerce
ELAN & E-Learning Academic Network Niedersachsen\\  %Einleitung, Vorgaben
\end{longtable}


% Weiter mit Abbildungen
\cleardoublepage
\addcontentsline{toc}{chapter}{Abbildungen}
\renewcommand{\listfigurename}{Abbildungen}
\listoffigures

% Schlie{\ss}lich Literatur
\cleardoublepage
\addcontentsline{toc}{chapter}{Literatur}
\bibliographystyle{alphadin}
\renewcommand{\bibname}{Literatur}
\bibliography{content/bibliographie}

% Und ganz am Ende der Index
\cleardoublepage
\addcontentsline{toc}{chapter}{Index}
\begin{theindex}
{\indexfrontskip\Large\sffamily\bfseries\hfill A\hfill}\nopagebreak
 
  \item Abbildungen\dotfill 2
  \item \"Anderungen\dotfill 2
  \item Anwendung\dotfill 1

  \indexspace
{\indexfrontskip\Large\sffamily\bfseries\hfill E\hfill}\nopagebreak
 
  \item Einbinden\dotfill 1
    \subitem Abbildungen\dotfill 2

  \indexspace
{\indexfrontskip\Large\sffamily\bfseries\hfill F\hfill}\nopagebreak
 
  \item Festlegungen\dotfill 7

  \indexspace
{\indexfrontskip\Large\sffamily\bfseries\hfill I\hfill}\nopagebreak
 
  \item Indexerstellung\dotfill 3
  \item Internationalisierung\dotfill 4

  \indexspace
{\indexfrontskip\Large\sffamily\bfseries\hfill L\hfill}\nopagebreak
 
  \item Literatur\dotfill 7
  \item Literaturreferenzen\dotfill 7

  \indexspace
{\indexfrontskip\Large\sffamily\bfseries\hfill P\hfill}\nopagebreak
 
  \item Paket\dotfill 1

  \indexspace
{\indexfrontskip\Large\sffamily\bfseries\hfill S\hfill}\nopagebreak
 
  \item Schriften\dotfill 5

  \indexspace
{\indexfrontskip\Large\sffamily\bfseries\hfill T\hfill}\nopagebreak
 
  \item Titelseite\dotfill 8

  \indexspace
{\indexfrontskip\Large\sffamily\bfseries\hfill V\hfill}\nopagebreak
 
  \item Verzeichnisreihenfolge\dotfill 7

  \indexspace
{\indexfrontskip\Large\sffamily\bfseries\hfill Z\hfill}\nopagebreak
 
  \item Zitate\dotfill 7

\end{theindex}


% Und schlie{\ss}lich noch die Versicherung f\"{u}r das Pr\"{u}fungsamt
\cleardoublepage
\versicherung{Oldenburg}

\end{document}
\end{Verbatim}
\caption{Anzupassende Pakete bzw. Verzeichnisbezeichner}\label{verzeichnisanpassung}
\end{figure}




\section{Einbetten s\"{a}mtlicher Schriften}\index{Schriften}\label{Schriften}
Bitte unbedingt folgende Dokumenteneinstellung beibehalten

\begin{verbatim}
\documentclass[11pt,a4paper]{book}
\end{verbatim}

Grunds\"{a}tzlich bietet es sich an, die Arbeit mit \texttt{pdflatex} direkt als PDF zu erzeugen. Bei der Verwendung von \texttt{pdflatex} k\"{o}nnen als Bildformate PDF, PNG oder JPG (Achtung: Die beiden letzteren sind nicht im Vektorformat und damit nicht gut f\"{u}r den Ausdruck geeignet.) f\"{u}r Bilder verwendet werden.

Es sollten aber s\"{a}mtliche verwendete Schriftarten in das
resultierende PDF-Do\-kument eingebettet werden. \texttt{pdflatex}
und \texttt{ps2pdf} sparen unter Umst\"{a}nden einige Schriften aus, sollten also nur mit Bedacht verwendet werden.

Eine M\"{o}glichkeit, dennoch s\"{a}mtliche Schriftarten in das
resultierende Dokument einzubetten, bietet eine Kombination aus
\texttt{dvips} und \texttt{ghostscript} an. Zun\"{a}chst das
resultierende DVI-Dokument \texttt{dissertation.dvi} in ein
passendes PS-Dokument umwandeln mit dem folgenden Befehl

\begin{verbatim}
dvips -P pdf -t A4 -z -Pdownload35 dissertation.dvi
\end{verbatim}

Die resultierende Datei \texttt{dissertation.ps} darf daraufhin
NICHT mit
\texttt{ps2pdf} sondern muss -- wie der folgende Befehl zeigt --
mit \texttt{ghostscript} in ein PDF umgewandelt werden

\begin{verbatim}
gs -dSAFER -dNOPAUSE -dBATCH -sDEVICE=pdfwrite \

   -sPAPERSIZE=a4 -dPDFSETTINGS=/printer \

   -dCompatibilityLevel=1.3 -dMaxSubsetPct=100 \

   -dSubsetFonts=true -dEmbedAllFonts=true \

   -sOutputFile=dissertation.pdf dissertation.ps
\end{verbatim}

Das hieraus resultierende PDF-Dokument \texttt{dissertation.pdf}
enth\"{a}lt schlie{\ss}lich sowohl das gew\"{u}nschte Format als auch
s\"{a}mtliche Schriften. Bei dem Umweg \"{u}ber EPS ist allerdings zu beachten, dass Vektorgraphiken nicht als PDF sondern als EPS-Dateien eingebunden werden m\"{u}ssen.

\chapter{Festlegungen der Abteilung}\index{Festlegungen}
\"{U}ber das reine Design hinaus sind nachfolgend Festlegungen zur
Struktur einer Arbeit (insbesondere Studentischer Arbeiten) der
Abteilung Informationssysteme  aufgef\"{u}hrt.



\section{Verzeichnisreihenfolge}\index{Verzeichnisreihenfolge}
Die Reihenfolge der Verzeichnisse soll dem vorliegenden Dokument
folgen, die Verzeichnisse dazu wie folgt angeordnet sein. Im
Inhaltsverzeichnis, das neuerdings nur noch "`Inhalt"' hei{\ss}t,
werden nur die nachfolgend fett dargestellten Punkte aufgef\"{u}hrt.

\begin{enumerate}
\item Zusammenfassung
\item Abstract
\item Inhalt (vormals Inhaltsverzeichnis)
\item \textbf{Der eigentliche Inhalt} mit laufenden Kapitelnummern
\item \textbf{Glossar} und Folgende jeweils ohne Kapitelnummer
\item \textbf{Abk\"{u}rzungen}
\item \textbf{Abbildungen}
\item \textbf{Literatur}
\item \textbf{Index}
\end{enumerate}


Hierbei wurde auf die traditionelle Verwendung der Begriffe
Inhaltsverzeichnis, Literaturverzeichnis und Abbildungsverzeichnis
verzichtet und stattdessen nur noch Inhalt, Literatur und
Abbildungen verwendet.


\section{Zitate und Literaturreferenzen}\index{Zitate}\index{Literaturreferenzen}\index{Literatur}
Im eigentlichen Text kann man verschiedene Zitate verwenden. Das
machen auch z.B. \cite{abel03}, \cite{appelrath06}, \cite{back01}
oder \cite{bakos91}. Den jeweiligen Zitaten soll dazu der
Bibliographiestil \texttt{alphadin} (mit kleinem "`a"') zugrunde
liegen. Dieser repr\"{a}sentiert einen genormten Zitierstil und wird
durch folgende Zeile im \LaTeX-Dokument eingestellt

\begin{verbatim}
\bibliographystyle{alphadin}
\end{verbatim}

Wer trotz des verk\"{u}rzten Zitierstils in ausf\"{u}hrlicher Weise auf
die jeweiligen Autoren hinweisen m\"{o}chte, hat die M\"{o}glichkeit, dies
z.B. folgenderma{\ss}en zu machen: "`wie auch von Appelrath, Boles,
Kleinefeld und andere in
\cite{appelrath06} beschrieben\ldots"'

Bei den Literaturangaben wird k\"{u}nftig auf die ISBN- bzw.
ISSN-Nummern verzichtet, da bei Dissertationen ohnehin davon
auszugehen ist, dass begutachtete bzw. verlegte Literatur
verwendet wird.


\section{Die offizielle Titelseite}\index{Titelseite}
Der \texttt{maketitle}-Befehl wurde \"{u}berschrieben. F\"{u}r die neue Titelseite
sind daher folgende Einstellungen zu machen

\begin{figure}[ht]
\centering
\begin{Verbatim}[label=dissertation.tex,numberblanklines=false,fontsize=\scriptsize,numbers=left,frame=single]
% Voreinstellungen f\"{u}r offizielle Titelseite
\title{Hier den Titel einf\"{u}gen}                   % Titel der Dissertation
\author{Dipl.-Inform. Vorname Nachname}           % Name des Autors
\arbeitstyp{Diplomarbeit}                         % oder Bachelorarbeit, Masterarbeit ...
\erstgutachter{Prof. Dr. Erster Gutachter}        % Name des Erstgutachters
\zweitgutachter{Prof. Dr.-Ing. Zweiter Gutachter} % Name des Zweitgutachters

% Einbinden/Anzeigen der Titelseite innerhalb von \begin/end{document}
\maketitle
\end{Verbatim}
\caption{Einstellungen f\"{u}r offizielle Titelseite}\label{titelseite}
\end{figure}



% Verzeichnisse am Ende, erst das Glossar
\addonchapter{Glossar} % Es soll auch Glossar hei{\ss}en
Nachfolgend sind noch einmal wesentliche Begriffe dieser Arbeit
zusammengefasst und erl�utert. Eine ausf�hrliche Erkl�rung findet
sich jeweils in den einf�hrenden Abschnitten sowie der jeweils
darin angegebenen Literatur. Das im Folgenden im Rahmen der
Erl�uterung verwendete Symbol \this bezieht sich jeweils auf den
im Einzelnen vorgestellten Begriff, das Symbol \siehe{} verweist
auf einen ebenfalls innerhalb dieses Glossars erkl�rten Begriff.

\begin{description}
\item[Auktion] Eine \this ist das im \siehe{E-Commerce} am
H�ufigsten eingesetzte Verfahren zur dynamischen
\siehe{Preisfindung}. Interessenten k�nnen dabei durch Abgabe von
Geboten Preis, Dauer und Gewinner beeinflussen. Bei einer offenen
\this sind Bieter, H�he der Gebote und der aktuelle Preis f�r
alle Teilnehmer sichtbar, bei der geschlossenen (sealed) \this
erfolgt nur eine interne Benachrichtigung. Die bekanntesten Typen
sind die traditionelle \siehe{Versteigerung} sowie die
\siehe{holl�ndische}, \siehe{umgekehrte} und \siehe{verdeckte} \this.

\item[Behaviorismus] Der \this ist eine \siehe{Lerntheorie}, die
davon ausgeht, dass Wissen als Struktur unabh�ngig vom
\siehe{Lernenden} existiert und dass sein Verhalten operant
konditioniert ist, d.h. dass es als Konsequenz aus anderen
Verhaltensweisen resultiert. Erfolgt eine positive Reaktion,
beh�lt der \siehe{Lernende} neu erlerntes Verhalten bei, negative
Reaktionen f�hren zu einer Verminderung dieses Verhaltens. Der
\siehe{Lehrende} bestimmt dabei das zu erlernende Wissen und
ist f�r die Steuerung des \siehe{Lernprozesses} zust�ndig.
\end{description}


% Dann die Abk\"{u}rzungen
\addonchapter{Abk�rzungen} % Die sollen auch Abk\"{u}rzungen hei{\ss}en
\begin{longtable}[ht]{ll}
ADDIE & Analysis, Design, Development, Implementation, Evaluation\\  %E-Learning
ADEPT & Advanced Decision Environment for Process Tasks\\  %Konzept
ADL & Advanced Distributed Learning Initiative\\  %E-Learning
AGB & Allgemeine Gesch�ftsbedingungen\\  %Konzept
AGOF & Arbeitsgemeinschaft Online Forschung\\  %Einleitung
AICC & Aviation Industry Computer Based Training Committee\\  %E-Learning
API & Application Programming Interface\\  %TEL, Konzept
%ARIADNE & Alliance of Remote Instructional Authoring and Distribution Networks for Europe\\  %E-Learning
ARIS & Architektur integrierter Informationssysteme\\  %Konzept
ASC & Accredited Standards Committee\\  %EC-Grundlagen
ASP & Application Service Providing\\  %Integration
ASTD & American Society for Training and Development\\  %E-Learning
AXIS & Apache eXtensible Interaction System\\  %Implementierung
B2B & Business-to-Business\\  %Konzept, Glossar
B2C & Business-to-Consumer\\  %Konzept, Glossar
BDSG & Bundesdatenschutzgesetz\\  %EC-Grundlagen
BGB & B�rgerliches Gesetzbuch\\  %Konzept
BITKOM & Bundesverband Informationswirtschaft, Telekommunikation und neue Medien\\  % Grundlagen E-Commerce
BMBF & Bundesministerium f�r Bildung und Forschung\\  %Einleitung
BME & Bundesverband Materialwirtschaft, Einkauf und Logistik\\  %EC-Grundlagen
BPEL4WS & Business Process Execution Language for Web Services\\  %Implementierung
BSCW & Basic Support for Cooperative Work\\  %E-Learning
BSI & Bundesamt f�r Sicherheit in der Informationstechnik\\ %Literatur
CAL & Computer Aided/Assisted Learning\\  %E-Learning
CBT & Computer Based Training\\  %E-Learning, Vorgaben
CC & Creative Commons\\  %Konzept
CD & Compact Disc\\  %E-Learning
CELab & Labor f�r Content Engineering\\  %TEL
CMI & Computer Managed Instruction\\  %E-Learning
CMS & Content Management System\\  %TEL, Konzept
CORBA & Common Object Request Broker Architecture\\  %Konzept
CPU & Central Processing Unit\\  %Konzept
CSCL & Computer Supported Collaborative Learning\\  %E-Learning
CSCW & Computer Supported Cooperative Work\\  %TEL
CSS & Customer Support Services\\  %Konzept
CRM & Customer Relationship Management\\  %Konzept
CUL & Computerunterst�tztes Lernen\\  %E-Learning
DBMS & Datenbankmanagementsystem\\  %Konzept
DCOM & Distributed Component Object Model\\  %Konzept
DFN & Deutsches Forschungsnetz\\  %E-Learning
DIN & Deutsches Institut f�r Normung\\  %Lernen, E-Learning, TEL
DREL & Digital Rights Expression Language\\  %Konzept
DRM & Digital Rights Management\\  %Konzept
DVD & Digital Video Disc\\  %E-Learning
E2B & Education-to-Business\\  %Integration
E2C & Education-to-Consumer\\  %Integration
E2E & Education-to-Education\\  %Integration
EAI & Enterprise Application Integration\\  %Konzept
EAN & European Article Numbering\\  %Grundlagen-EC-Standards
EBPP & Electronic Bill Presentment and Payment\\  %Konzept
ebXML & Electronic Business Extensible Markup Language\\ %EC-Grundlagen
ECA & Event-Condition-Action\\  %Vorgaben
EC & Electronic Cash\\  %EC-Grundlagen, Konzept
ECC & E-Learning Courseware Certification\\  %E-Learning
EDI & Electronic Data Interchange\\  %EC-Grundlagen
EDIFACT & EDI for Administration, Commerce and Transport\\  %Konzept
EFQM & European Foundation for Quality Management\\  %TEL
EGBGB & Einf�hrungsgesetz zum B�rgerlichen Gesetzbuch\\  %Konzept
EITO & European Information Technology Observatory\\  % Grundlagen E-Commerce
ELAN & E-Learning Academic Network Niedersachsen\\  %Einleitung, Vorgaben
\end{longtable}


% Weiter mit Abbildungen
\cleardoublepage
\phantomsection % generiert Anker f\"{u}r \addcontentsline
\addcontentsline{toc}{chapter}{Abbildungen} % Sowohl im Inhaltsverzeichnis als auch als
\renewcommand{\listfigurename}{Abbildungen} % \"{U}berschrift als "Abbildungen" ohne -verzeichnis
\listoffigures

% Schlie{\ss}lich Literatur
\cleardoublepage
\phantomsection % generiert Anker f\"{u}r \addcontentsline
\addcontentsline{toc}{chapter}{Literatur} % Soll als "Literatur" auftauchen
\bibliographystyle{alphadin}                 % Unser Standard-Bibliopgraphiestyle
\renewcommand{\bibname}{Literatur}        % Auch als \"{U}berschrift soll "Literatur" erscheinen
\bibliography{content/bibliographie}            % Literaturverzeichnis einbinden

% Und ganz am Ende der Index
\cleardoublepage
\phantomsection % generiert Anker f\"{u}r \addcontentsline
\addcontentsline{toc}{chapter}{Index} % Der auch Index hei{\ss}en soll
\begin{theindex}
{\indexfrontskip\Large\sffamily\bfseries\hfill A\hfill}\nopagebreak
 
  \item Abbildungen\dotfill 2
  \item \"Anderungen\dotfill 2
  \item Anwendung\dotfill 1

  \indexspace
{\indexfrontskip\Large\sffamily\bfseries\hfill E\hfill}\nopagebreak
 
  \item Einbinden\dotfill 1
    \subitem Abbildungen\dotfill 2

  \indexspace
{\indexfrontskip\Large\sffamily\bfseries\hfill F\hfill}\nopagebreak
 
  \item Festlegungen\dotfill 7

  \indexspace
{\indexfrontskip\Large\sffamily\bfseries\hfill I\hfill}\nopagebreak
 
  \item Indexerstellung\dotfill 3
  \item Internationalisierung\dotfill 4

  \indexspace
{\indexfrontskip\Large\sffamily\bfseries\hfill L\hfill}\nopagebreak
 
  \item Literatur\dotfill 7
  \item Literaturreferenzen\dotfill 7

  \indexspace
{\indexfrontskip\Large\sffamily\bfseries\hfill P\hfill}\nopagebreak
 
  \item Paket\dotfill 1

  \indexspace
{\indexfrontskip\Large\sffamily\bfseries\hfill S\hfill}\nopagebreak
 
  \item Schriften\dotfill 5

  \indexspace
{\indexfrontskip\Large\sffamily\bfseries\hfill T\hfill}\nopagebreak
 
  \item Titelseite\dotfill 8

  \indexspace
{\indexfrontskip\Large\sffamily\bfseries\hfill V\hfill}\nopagebreak
 
  \item Verzeichnisreihenfolge\dotfill 7

  \indexspace
{\indexfrontskip\Large\sffamily\bfseries\hfill Z\hfill}\nopagebreak
 
  \item Zitate\dotfill 7

\end{theindex}
                     % Index einbinden, vorher aber mit makeindex erzeugen!

% Und schlie{\ss}lich noch die Versicherung f\"{u}r das Pr\"{u}fungsamt, Parameter ist der Ort der Unterschrift
\cleardoublepage
\versicherung{Oldenburg}

\end{document}
