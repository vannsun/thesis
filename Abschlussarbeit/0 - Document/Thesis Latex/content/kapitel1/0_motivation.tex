\chapter{Einleitung} \index{Einleitung}
\section{Motivation}

Die Struktur des Studienplans eines akademischen Bachelor an der Carl von Ossietzky Universit�t Oldenburg besteht aus vier verschiedenen Modulkategorien. Zum einen aus Basis- und Aufbaumodulen, die zu dem Pflichtbereich geh�ren und die wichtige Grundlage des Studiums vermitteln; zum zweiten aus Akzentssetzungsmodule, die Studierenden eine individuelle Ausrichtung geben k�nnen. Zus�tzlich gibt es Professionalisierungs- und Praxismodule, die den Erwerb berufsbezogener und praktischer Kenntnisse stellen. Zul�sst muss ein Abschlussmodul abgeschlossen werden.
Master of Education und Master sind anderes aufgebaut aber auch f�r sie gibt es Abschlussarbeit.\\
Das Abschlussmodul ist eine vertiefende Pr�fungsleitung, die von den Studenten als Voraussetzung f�r die Qualifikation des Abschlusses entwickelt wurde\cite{BScInf:2020}.\\

Bei der Anfertigung der Abschlussarbeit muss die\textbackslash der Studierende zun�chst ein Thema f�r ihre\textbackslash seine Forschungsarbeit finden, welches f�r sein Studienprogramm relevant sein muss. Es ist m�glich, dass Studierende in Absprache mit einer Gutachterin bzw. einem Gutachter selbst ein Thema f�r eine Abschlussarbeit w�hlen k�nnen, oder dass, es durch eine Einrichtung au{\ss}erhalb der Universit�t ausgef�hrt werden kann\cite{Boles:2015}.\\

Das Department f�r Informatik ist in vier Fachrichtungen aufgegliedert: Theoretische Informatik, Praktische Informatik, Angewandte Informatik und Technische Informatik, und jede davon ist nochmal in verschiedene Vertiefungsrichtungen unterteil. Insgesamt gibt es zur Zeit siebzehn Abteilungen\cite{SpeInf:2020}. Jede Abteilung setz sich aus Professoren und wissenschaftliche Mitarbeiter zusammen, die eine Forschungsgruppe bilden und f�r die von ihr durchgef�hrten Projekte verantwortlich sind.\\
F�r Abschlussprojekte, die an der Universit�t entwickelt wurden, sollte jede Abteilung des Departments f�r Informatik auf einer Abteilungs-eigenen Website, aktuelle Themen f�r Abschlussarbeiten ver�ffentlichen oder beschreiben, wie Studierende in der entsprechenden Abteilung ein Thema erh�lt\cite{Boles:2015}.\\

Der Zugang zu Informationen �ber die Abschlussthemen sollte von dem Department f�r Informatik gew�hrleistet werden, indem es eine standardisierte Form der Suche anbietet, welche den Universit�tsmitarbeitern erm�glicht, die ver�ffentlichten Informationen effizient zu aktualisieren, und Studierende erm�glicht, ihre pers�nlichen Interessen zu ber�cksichtigen. Die aktuelle L�sung, alle Abschlussthemen auf einer zentralen Website zu verkn�pfen, hat sich nicht bew�hrt und sollte durch diese Arbeit abgel�st werden.

Die Suche nach Abschlussthemen durch Studierende ist nicht einfach.  Das umfangreiche Angebot an Spezialisierungen bzw. Abteilungen des Departments f�r Informatik erm�glicht es den Studenten, sich mit dem Bereich zu befassen, der sie am meisten anspricht. Es gibt jedoch zahlreiche Forschungsthemen, die sogar zu verschiedenen Studienbereichen geh�ren k�nnen, wie bei Projekten der Abteilung Didaktik der Informatik - Fachgebiet Angewandte Informatik, wo Kenntnisse der Mikrorobotik / Regelungstechnik und Softwaretechnik, die Teil der Fachrichtungen Technische Informatik bzw. Praktische Informatik sind, erforderlich sein k�nnen.\\