\section{Struktur des Departements f�r Informatik}\index{Struktur}\index{Departament}
Das Department geh�rt zur Fakult�t II - Informatik, Wirtschafts- und Rechtswissenschaften, es besteht derzeit aus

\begin{itemize}
	\item[--] 17 ProfessorInnen und ihren Abteilungen17 ProfessorInnen und ihren Abteilungen
	\item[--] ca.1400 Studierenden, ca. 50 wissenschaftlichen Mitarbeitern
	\item[--] drei weiteren f�r das Department zentralen Einrichtungen, die Service-Aufgaben in Lehre und Forschung �bernehmen
	\item[--] sowie einer Departementsverwaltung
\end{itemize}

Das Department f�r Informatik ist in vier Fachrichtungen aufgegliedert und jede davon ist nochmal in verschiedene Vertiefungsrichtungen unterteilt.

\begin{itemize}
	\item Theoretische Informatik
	\begin{itemize}
		\item Entwicklung korrekter Systeme
		\item Parallele Systeme
	\end{itemize}
	\item Praktische Informatik
	\begin{itemize}
		\item Medieninformatik und Multimedia-Systeme
		\item Computational Intelligence
		\item Systemsoftware und verteilte Systeme
		\item Softwaretechnik
		\item Informationssysteme
	\end{itemize}
	\item Angewandte Informatik
	\begin{itemize}
		\item Didaktik der Informatik
		\item Wirtschaftsinformatik Systemanalyse und Optimierung
		\item Intelligente Transportsysteme
		\item Energieinformatik
		\item Wirtschaftsinformatik Very Large Business Applications
		\item Digitalisierte Energiesysteme
	\end{itemize}
	\item Technische Informatik
	\begin{itemize}
		\item Sicherheitskritische eingebettete Systeme
		\item Mikrorobotik und Regelungstechnik
		\item Hybride Systeme
		\item Eingebettete Hardware-/Software-Systeme
	\end{itemize}
\end{itemize}

Jede Abteilung setz sich aus Professoren und wissenschaftliche Mitarbeiter zusammen, die eine Forschungsgruppe bilden und f�r die von ihr durchgef�hrte Projekte verantwortlich sind.
F�r Abschlussprojekte, die an der Universit�t entwickelt wurden, sollte im Moment jede Abteilung des Departments f�r Informatik auf einer Abteilungs-eigenen Webseite, aktuelle Themen f�r Abschlussarbeiten ver�ffentlicht oder beschreiben, wie Studierende in der entsprechenden Abteilung ein Thema erhalten kann.