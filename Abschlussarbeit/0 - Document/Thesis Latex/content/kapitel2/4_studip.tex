\section{Stud.IP}\index{Stud.IP}
Stud.IP ist ein Campus-Lernmanagementsystem und steht f�r "Studienbegleitender Internetsupport von Pr�senzlehre" \cite{stIP20}. Alle Unterrichtsprozesse werden von Stud.IP unterst�tzt, indem die Verwaltung der Prozesse von der Planung der Unterrichten und Zuordnung der Unterrichtsr�ume bis zum Qualit�tsmanagement konsolidiert wird. 

Stud.IP besteht aus einer Kernumgebung, die eine Vielzahl von Tools anbietet, von der kollaborativen Texterstellung �ber Lernmodule bis hin zum Dokumentenmanagement, es werden im Kern Basisfunktionalit�ten bereitgestellt. Der Kern von Stud.IP kann weiterentwickelt werden um die Funktionalit�ten des Systems zu erweitern.

Stud.IP und die meisten der verf�gbare Plugins sind Open Source Software und stehen unter der General Public License. Der Kern ist nachhaltig und wird kontinuierlich weiterentwickelt.

Die Universit�t Oldenburg implementierte zum Wintersemester 2003/2004 das Open-Source-LMS Stud.IP und wechselte vom kommerziellen LMS Blackboard \cite{appelrath06}. Einer der Vorteile des Open-Source-Ansatzes besteht darin, dass lokale Anpassungen an den Anforderungen der Lehrenden und Studierenden vorgenommen werden k�nnen.
F�r die Universit�t Oldenburg wurden einige Module eingebaut wie die Gruppenverwaltung in Veranstaltungen oder das "Schwarzes Brett" Plugin.

\subsection{Softwarearchitektur}

Stud.IP ist eine PHP-Softwareanwendung, die eine MySQL-Datenbank verwendet. Als Webserver wird der Apache unterst�tzt. 

Wer mitentwickeln will, braucht also vor allem PHP-Kenntnisse, muss sich etwas mit SQL auskennen und ein bisschen �ber Apache-Konfiguration wissen. Und, wie immer bei Webanwendungen: Alle Ausgaben geschehen in HTML, formatiert duch CSS. Einige Funktionen verwenden zudem XML als Zwischenformat, Javascript und AJAX sind ebenfalls an vielen Stellen pr�sent. Wenn all das keine Fremdw�rter f�r dich sind, bis du gut ger�stet.

%\subsection{Trails}
\subsection{Stud.IP Datenbank}
Bei der Stud.IP-Datenbank handelt es sich um eine MariaDB-Datenbank. Das Datenbankschema ist �ber die Jahre erweitert worden, es sind keine Informationen �ber die Struktur zu finden.