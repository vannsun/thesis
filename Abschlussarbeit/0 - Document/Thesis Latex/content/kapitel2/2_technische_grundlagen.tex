\section{Technische Grundlagen der Software}\index{Technische}\index{Grundlagen}
Dieses Unterkapitel zielt darauf ab, den \textbf{Begriff Weiterentwicklung von Softwaresystemen} n�her zu betrachten. //TODO: Beschreibung Kapitel

\subsection{Weiterentwicklung von Softwaresystemen}\index{Weiterentwicklung}\index{Softwaresysteme}

\subsection{LimeSurvey - Umfrage Tool}\index{Weiterentwicklung}\index{Softwaresysteme}


\subsection{Objektorientierte Programmierung}\index{Objektorientierte Programmierung}\index{Softwaresysteme}


\subsection{Design Patterns}\index{Design Patterns}


\subsubsection{Model View Controller}\index{Objektorientierte Programmierung}\index{Softwaresysteme}


\subsection{Entwicklungssystem mit XAMPP}\index{XAMPP}
XAMPP ist ein Open-Source-Softwarepaket, der das Installieren und Konfigurieren des Webservers Apache mit dem Datenbankverwaltungssystem MariaDB und die Skriptsprachen PHP und Perl erm�glicht. 
Der Name ist ein Akronym: \textbf{X} (f�r eines der verschiedenen Betriebssysteme cross-platform), \textbf{A}pache, \textbf{M}ariaDB / MySQL, \textbf{P}HP, \textbf{P}erl. Ab Version 5.6.15 hat XAMPP die MySQL-Datenbank in MariaDB ge�ndert, eine GPL-lizenzierte Abzweigung von MySQL.

Die Entwicklungsumgebung wurde durch Installation von Version 7.3.23 von XAMPP f�r das Betriebssystems Windows 10 konfiguriert.

Bei der Durchf�hrung dieses Projekts wurden Software-Artefakte mit MariaDB und PHP generiert.

\subsection{MariaDB}\index{MariaDB}\index{MySQL}


Die Entwicklungsgeschichte von MySQL geht bis ins Jahr 1979 zur�ck, bis es schlie�lich von der Firma Oracle �bernommen wurde. 

Die Verwendungsspektrum von MariaDB ist gro�, da es alle vorteile von kommerziellen Anbietern verenint. MariaDB setzt dabei auf die klassische Cliente-Server-Architektur, in der ein zentraller Datenbankserver die Daten verwaltet, worauf ein Datenbank-Client �ber das netzwerzugreifen kann. Weitere Vorteile sind die meherbenutzerf�higkeit ohne Performance-Einbu�en, die l


\subsection{PHP}\index{PHP}
PHP existiert bereits seit dem Jahre 1994 und war anfangs sehr beschr�nkt in seinem Funktionsumfang.
Zun�chst war es lediglich f�r die Dynamisierung von statischen Webseiteninhalten geeignet.
Erst im Jahre 1998 wurde die Funktionalit�t durch Version 3.0 deutlich erweitert, da von nun an
verschiedene Datenbanksysteme angebunden werden konnten.

Im Laufe der Zeit wurde PHP stetig weiterentwickelt und steht zum gegenw�rtigen Zeitpunkt in der
Version 5.5.15 zur Verf�gung.

Zwar wird PHP von vielen Entwicklern aufgrund der Einfachheit sehr kritisch bewertet, es steht
gegen�ber den h�heren Programmiersprachen, wie Java oder C++, in nichts nach. Dem Entwickler /
Entwicklerin stehen beispielsweise Konzepte wie Kapselung, Vererbung, Namensr�ume genauso zur
Verf�gung wie Closures, womit Funktionalit�ten in anonyme Methoden gekapselt werden k�nnen. Somit ist das objektorientierte Programmieren im vollen Umfang m�glich.

Ein weiterer Vorteil von PHP ist die Plattformunabh�ngigkeit, wodurch es auf den meisten Betriebssystemen installiert und genutzt werden kann. Anzumerken ist jedoch auch, dass PHP nur eine Skriptsprache ist und erst zur Laufzeit vom PHP-Interpreter ausgef�hrt wird. Das hat zwar den Vorteil, dass eine �bersetzung in Maschinencode nicht n�tig ist, ist auch ein Nachteil bei zeitkritischen Aufgaben.

Ein weiterer, nicht neuer Ansatz soll die Vielseitigkeit von PHP noch einmal verdeutlichen: Facebook
transformiert den PHP-Quellcode in ein �quivalentes C++ Konstrukt und �bersetzt dies mit einem geeigneten.

Compiler in Maschinencode. Der Vorteil dabei ist, dass jene Anwendung einerseits schneller
ist und andererseits die Webserver entlastet.
PHP erfreut sich gro�er Beliebtheit, da mittlerweile mehr als 240 Millionen Webseiten PHP f�r die
Erstellung von dynamischen Webseiten verwenden.