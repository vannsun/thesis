\section{Einfürung}
Die Struktur des Studienplans eines akademischen Bachelorprogramms an der Carl von Ossietzky Universität Oldenburg besteht aus vier verschiedenen Modulkategorien. Zum einen aus Basis- und Aufbaumodulen, die zu dem Pflichtbereich gehören und die wichtige Grundlage des Studiums vermitteln; zum zweiten aus Akzentssetzungsmodule, die Studierenden eine individuelle Ausrichtung geben können. Zusätzlich gibt es Professionalisierungs- und Praxismodule, die den Erwerb berufsbezogener und praktischer Kenntnisse stellen. Zulässt muss ein Abschlussmodul abgeschlossen werden. Master of Education und Master sind anderes aufgebaut aber auch für sie gibt es Abschlussarbeit.
Das Abschlussmodul ist eine vertiefende Prüfungsleitung, die von den Studenten als Voraussetzung für die Qualifikation des Abschlusses entwickelt wurde\cite{BScInf:2020}.\\

Bei der Anfertigung der Abschlussarbeit muss die\textbackslash der Studierende zunächst ein Thema für ihre\textbackslash seine Forschungsarbeit finden, welches für sein Studienprogramm relevant sein muss. Es ist möglich, dass Studierende in Absprache mit einer Gutachterin bzw. einem Gutachter selbst ein Thema für eine Abschlussarbeit wählen können, oder dass, es durch eine Einrichtung au{\ss}erhalb der Universität ausgeführt werden kann\cite{Boles:2015}.\\

Das Department für Informatik ist in vier Fachrichtungen aufgegliedert: Theoretische Informatik, Praktische Informatik, Angewandte Informatik und Technische Informatik, und jede davon ist nochmal in verschiedene Vertiefungsrichtungen unterteil. Insgesamt gibt es siebzehn Abteilungen\cite{SpeInf:2020}. Jede Abteilung setz sich aus Professoren und wissenschaftliche Mitarbeiter zusammen, die eine Forschungsgruppe bilden und für die von ihr durchgeführten Projekte verantwortlich sind.
Für Abschlussprojekte, die an der Universität entwickelt wurden, jede Abteilung des Departments für Informatik sollte, auf einer Abteilungs-eigenen Website, aktuelle Themen für Abschlussarbeiten veröffentlichen oder beschreiben, wie Studierende in der entsprechenden Abteilung ein Thema erhält\cite{Boles:2015}.\\

Der Zugang zu Informationen über die Abschlussthemen sollte von dem Department für Informatik gewährleistet werden, indem es eine standardisierte Form der Suche anbietet, welche den Universitätsmitarbeitern ermöglicht, die veröffentlichten Informationen effizient zu aktualisieren, und Studierende ermöglicht, ihre persönlichen Interessen zu berücksichtigen.

\newpage
\subsection{Zielsetzung}
Diese Bachelorarbeit beschäftigt sich gezielt mit der Themenfindung und Verwaltung von Abschlussarbeiten. In diesem Zusammenhang sollte klargestellt werden, welche Kriterien sind für die Suche nach einem Forschungsthema relevant und unter welchen Parametern kann der aktuelle Prozess der Veröffentlichung und Zuordnung von Abschlussarbeitsthemen optimiert werden. Diese Informationen werden für die Entwicklung und Implementierung einer Softwareanwendung berücksichtigt, mit der Studierende nach Themen von Abschlussarbeiten suchen und Mitarbeiter des Fachbereichs Informatik diese zuweisen und verfolgen können.
