\section{Stand der Forschung}
Im Bereich der Durchführung eines Studienprojekts wurde 2009\cite{Watat:2019} ein Lösungsvorschlag für das Problem der Suche nach Forschungsprojekten am DfI vorgelegt. Dieser Vorschlag wurde als Webanwendung konzipiert, die der Website der Universität hinzugefügt wurde. Dort müssen die Mitarbeiter der Abteilung Informationen aus Forschungsprojekten eingeben, die sowohl von Universitätsmitarbeitern als auch von externen Benutzern konsultiert werden können.
Die Webanwendung wurde mit HTML, CSS und Java sowie mit der MySQL-Datenbank programmiert. Die Suche erfolgt anonym, es erfolgt keine Authentifizierung.\\

Einige der Themen für Forschungsprojekte, die als Abschlussprojekte abgeschlossen werden können, werden derzeit auf der Website der Universität veröffentlicht. Die Informationen können öffentlich eingesehen werden, und da keine Authentifizierung vorliegt, ist es nicht möglich zu wissen, wie viele Personen an den Themen interessiert sind, oder zusätzliche Ma{\ss}nahmen zur Konsultation durchzuführen.\\

Die Universität nutzt die Stud.IP-Arbeitsumgebung, um die interne Kommunikation zwischen Fakultät und Studierenden zu verwalten. Dafür arbeiten die für die Verwaltung der Plattform zuständigen Mitarbeiter ständig an Tools und Funktionen, die die Benutzerinteraktion verbessern.
Stud.IP ist eine kostenlose, Open Source Softwareplattform, deren Hauptprogrammiersprache PHP ist\cite{SIPPHP:2020}. Unter diesen Merkmalen ist es möglich, die Software- und Programm-Widgets-Anwendungen als Ergänzung zur Hauptplattform herunterzuladen.\\
Als Open-Source-Software ist Stud.IP lizenzkostenfrei\cite{SIPOS:2020}, d.h. jeder kann sich die Software herunterladen, installieren und unbegrenzt nutzen. Entwickelt wird die Software von der Stud.IP-CoreGroup, der aktiven Entwicklungsgemeinschaft, einer Gemeinschaft von Betreibereinrichtungen, der data-quest GmbH sowie dem Hochschulverein ELAN e.V.\\

In letzter Zeit wurden Funktionen für die Stud.IP-Umgebung der Universität aufgenommen, mit denen Fragebögen (VIPs) und Umfragen (Stoodle) offen oder anonym durchgeführt werden können und deren Informationen zur Erstellung von Statistiken verwendet werden können. Diese Werkzeuge anderseits werden verwendet um die Anforderungen Studierenden und Mitarbeitern zu erheben, was zeigt, dass eine Ergänzung von Plugins möglich ist.