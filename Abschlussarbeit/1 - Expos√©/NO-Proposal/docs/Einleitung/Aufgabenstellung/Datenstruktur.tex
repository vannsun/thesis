\subsubsection{Vorlaufige Datenstruktur}
Eine Fakult\"at kann mehrere Departments haben, jedes Department kann aber nur zu einer Fakult\"at geh\"oren und es kann verschiedene Fachrichtungen besitzen.
Jede Fachrichtung kann nur zu einem Department zugeordnet werden und kann in mehrere Abteilungen unterteilt werden, wobei eine Abteilung nur zu einer Fachrichtung geh\"oren kann.
Mehrere Betreuer k\"onnen einer Abteilung zugeh\"orig sein aber ein Betreuer ist jeweils nur einer Abteilung zugeordnet.

Ein Thema besteht aus einem Identifikator, einem Namen, einer Beschreibung und ein Ver\"offentlichungsdatum. Es wird in einer Sprache geschrieben und wird einer Forschungsart zugeordnet: praktisch oder theoretisch. Ebenso besitzt das Thema einen Wert f\"ur den Status, wie zum Beispiel "reserviert" oder "in Bearbeitung". Au{ss}erdem werden f\"ur das Thema verschiedenen Kompetenzen vorausgesetzt, welche wiederum Voraussetzungen mehrerer Themen sein k\"onnen.

Sollte sich eine Studierende f\"ur ein Thema interessieren, kann sie es reservieren. Daraufhin kann ihr das Thema von einem Dozenten zugeordnet werden.

Die vorl\"aufige Entit\"aten und ihren Beziehungen werden unten vorgestellt: