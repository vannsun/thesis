Nachfolgend sind noch einmal wesentliche Begriffe dieser Arbeit
zusammengefasst und erl�utert. Eine ausf�hrliche Erkl�rung findet
sich jeweils in den einf�hrenden Abschnitten sowie der jeweils
darin angegebenen Literatur. Das im Folgenden im Rahmen der
Erl�uterung verwendete Symbol \this bezieht sich jeweils auf den
im Einzelnen vorgestellten Begriff, das Symbol \siehe{} verweist
auf einen ebenfalls innerhalb dieses Glossars erkl�rten Begriff.

\begin{description}
\item[Auktion] Eine \this ist das im \siehe{E-Commerce} am
H�ufigsten eingesetzte Verfahren zur dynamischen
\siehe{Preisfindung}. Interessenten k�nnen dabei durch Abgabe von
Geboten Preis, Dauer und Gewinner beeinflussen. Bei einer offenen
\this sind Bieter, H�he der Gebote und der aktuelle Preis f�r
alle Teilnehmer sichtbar, bei der geschlossenen (sealed) \this
erfolgt nur eine interne Benachrichtigung. Die bekanntesten Typen
sind die traditionelle \siehe{Versteigerung} sowie die
\siehe{holl�ndische}, \siehe{umgekehrte} und \siehe{verdeckte} \this.

\item[Behaviorismus] Der \this ist eine \siehe{Lerntheorie}, die
davon ausgeht, dass Wissen als Struktur unabh�ngig vom
\siehe{Lernenden} existiert und dass sein Verhalten operant
konditioniert ist, d.h. dass es als Konsequenz aus anderen
Verhaltensweisen resultiert. Erfolgt eine positive Reaktion,
beh�lt der \siehe{Lernende} neu erlerntes Verhalten bei, negative
Reaktionen f�hren zu einer Verminderung dieses Verhaltens. Der
\siehe{Lehrende} bestimmt dabei das zu erlernende Wissen und
ist f�r die Steuerung des \siehe{Lernprozesses} zust�ndig.
\end{description}
